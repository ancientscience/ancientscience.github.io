% spine = 0.002252 * 105 = 0.23646
% Minimum Cover Width: Bleed + Back Cover Trim Size + Spine Width + Front Cover Trim Size + Bleed =  0.125" + 6" + 0.23646" + 6" + .125" = 12.48646''
%  Minimum Cover Height: Bleed + Book Height Trim Size + Bleed = 0.125" + 9" + .125" = 9.25"
\documentclass[12pt, svgnames, dvipsnames]{article}
%\PassOptionsToPackage{cmyk}{xcolor}

\usepackage{comment, calligra}
\usepackage{amsmath, etoolbox, amssymb, manfnt}
\usepackage{pifont}
\usepackage[paperheight=9.25in,paperwidth=6.3in,margin=0in]{geometry}
%\usepackage{pdfpages}
\usepackage{tikz, comment}
\usepackage{fontspec}
%\usepackage{dtklogos}
\usepackage{varwidth} % hyphenate texts in TikZ Node (back cover)
\usetikzlibrary{fadings, shadings, shadows, patterns}


\usepackage[]{minted}

\renewcommand{\theFancyVerbLine}{\sffamily
\textcolor[rgb]{0.5,0.5,1.0}{\scriptsize
\oldstylenums{\arabic{FancyVerbLine}}}}


\newmintedfile[newcppfile]{cpp}
{
frame=none, linenos=false, samepage=false, numbersep=5pt, fontfamily=courier, 
fontseries=m, fontsize=\tiny, baselinestretch=0.1
}


\newminted[newcpp]{cpp}
{
frame=none, linenos=false, samepage=false, numbersep=7pt, fontfamily=courier, 
fontseries=m, fontsize=\tiny, baselinestretch=0.1
}


\newcommand{\itemcolor}[1]{% Update list item colour
  \renewcommand{\makelabel}[1]{\color{#1}\hfil ##1}}

\tikzset{
    shade border west to east/.style args={#1 to #2}{
        preaction={draw, very thick, path fading=east, #1},
        preaction={draw, very thick, path fading=west, #2}
    },
    shade fill west to east/.style args={#1 to #2}{
        left color=#1,
        right color=#2
    }
}

\def\nodeshadowed[#1]#2;{\node[scale=2,above,#1]{#2};\node[scale=2,
        above,#1,yscale=-1,scope fading=south,opacity=0.4]{#2};}
\def\nodeshadow[#1]#2;{\node[text=black, scale=1,above,#1]{#2};\node[text=black, scale=1, above,#1,yscale=-1,scope fading=south,opacity=1.0]{#2};}
        

 \special{pdf:docinfo <<
        /Title (C++14 FAQs)
        /Subject (C++)
        /Author (Chandra Shekhar Kumar)
        /Keywords ()
        /Trapped /False
        /GTS_PDFXVersion (PDF/X-1a:2003)
        /GTS_PDFXConformance (PDF/X-1a:2003)
     >>}

\special{pdf:put @catalog <<
	/OutputIntents [ <<
	/Info (none)
	/Type /OutputIntent
	/S /GTS_PDFX
	/OutputConditionIdentifier (createspace.com)
	/RegistryName (https://www.createspace.com/)
	>> ] >>
}

\colorlet{MyColor}{Maroon} 
\colorlet{TextColor}{White} 
\colorlet{BgTextColor}{LightCyan} 

\begin{document}
\pagestyle{empty}


\begin{tikzpicture}[remember picture,overlay]

\begin{scope}[shift={(current page.center)}]


% render background of front page
%\node[shading=ball, ball color=MyColor, minimum width=6.15in, minimum height=9.25in,text=white,ultra thick] at(3.2in,0) {};
%\node[shading=ball, ball color=MyColor, minimum width=6.3in, minimum height=9.25in,text=white,ultra thick] at(0in,0) {};
%\node[fill=MyColor, minimum width=6.3in, minimum height=9.25in,text=white,ultra thick] at(0in,0) {};

\node[shading=ball, ball color=MyColor, align=left, color=TextColor, text opacity=1.0, rounded corners=5ex,ultra thick] at(0.0in,5in) 
{
\tiny
\begin{varwidth}{16cm}
%\begin{comment}
\begin{newcpp}
#include <type_traits>
#include <chrono>
#include <tuple>

template<class T1, class T2 = T1>
inline T1 exchange(T1 & obj, T2 && new_value)
{
    T1 old_value = std::move(obj);

    obj = std::forward<T2>(new_value);

    return old_value;
}

struct increment_me 
{
    template<typename T>
    auto operator()(T t) const 
        -> decltype(++t)
    { 
        return ++t; 
    }
};

template<typename T=double>
constexpr T pi = T(3.1415926535897932385);
 
auto area = []( auto c ) 
{
    using T = typename decltype(c)::value_type;
    return pi<T> * c.radius * c.radius;
};

template<typename Array, size_t... I>
auto array_to_tuple_(const Array& a, std::index_sequence<I...>)
    -> decltype(std::make_tuple(a[I]...))
{
    return std::make_tuple(a[I]...);
}


template<typename T, std::size_t N, typename Indices = std::make_index_sequence<N>>
auto array_to_tuple(const std::array<T, N>& a)
    -> decltype(array_to_tuple_(a, Indices()))
{
    return array_to_tuple_(a, Indices());
}

template <class T>
struct C
{
    T t_;

    template <class U, class V = typename std::enable_if
                    <
                        !std::is_lvalue_reference<U>::value
                    >::type>
        C(U&& u) : t_(std::forward<T>(std::move(u).get())) {}
};

\end{newcpp}
%\end{comment}

\newcppfile[firstline=1, lastline=86]{make_integer_sequence.hpp}

\begin{newcpp}

template<class T1, class T2 = T1>
inline T1 exchange(T1 & obj, T2 && new_value)
{
    T1 old_value = std::move(obj);

    obj = std::forward<T2>(new_value);

    return old_value;
}

struct increment_me 
{
    template<typename T>
    auto operator()(T t) const 
        -> decltype(++t)
    { 
        return ++t; 
    }
};


template<typename T=double>
constexpr T pi = T(3.1415926535897932385);
 
auto area = []( auto c ) 
{
    using T = typename decltype(c)::value_type;
    return pi<T> * c.radius * c.radius;
};


template<typename Array, size_t... I>
auto array_to_tuple_(const Array& a, std::index_sequence<I...>)
    -> decltype(std::make_tuple(a[I]...))
{
    return std::make_tuple(a[I]...);
}


template<typename T, std::size_t N, typename Indices = std::make_index_sequence<N>>
auto array_to_tuple(const std::array<T, N>& a)
    -> decltype(array_to_tuple_(a, Indices()))
{
    return array_to_tuple_(a, Indices());
}


template <class T>
struct C
{
    T t_;

    template <class U, class V = typename std::enable_if
                    <
                        !std::is_lvalue_reference<U>::value
                    >::type>
        C(U&& u) : t_(std::forward<T>(std::move(u).get())) {}
};


int x;                              

struct A 
{
    // constexpr constructor
    constexpr A(bool b) : m(b?42:x) { }
    int m;
};


constexpr int v = A(true).m;        
                                    
constexpr int w = A(false).m; 



#include <utility>
#include <type_traits>

// generates std::size_t : 0, 1, 2, 3
using indices = 
    std::index_sequence_for<char, int, std::size_t, unsigned long>;

static_assert(std::is_same<
              indices,
              std::make_index_sequence<4>
             >::value, "");


struct A{};
struct B{};

static_assert(std::is_same<
    std::index_sequence_for<A, B>,
    std::index_sequence<0, 1>
>::value, "");



template< class T >
    using void_t = void;
    //using void_t = std::conditional_t<true, void, T>;


template< class, class T = void >
struct has_type_member : std::false_type 
{ };


template< class T >
struct has_type_member<T, void_t<typename T::type>> : std::true_type
{ };


template<typename Array, size_t... I>
auto array_to_tuple_(const Array& a, std::index_sequence<I...>)
    -> decltype(std::make_tuple(a[I]...))
{
    return std::make_tuple(a[I]...);
}


template<typename T, std::size_t N, typename Indices = std::make_index_sequence<N>>
auto array_to_tuple(const std::array<T, N>& a)
    -> decltype(array_to_tuple_(a, Indices()))
{
    return array_to_tuple_(a, Indices());
}


template <class T>
struct C
{
    T t_;

    template <class U, class V = typename std::enable_if
                    <
                        !std::is_lvalue_reference<U>::value
                    >::type>
        C(U&& u) : t_(std::forward<T>(std::move(u).get())) {}
};
\end{newcpp}

\end{varwidth}
};

% design front cover
% Title
\node[scale=2.6, text=YellowOrange, at={(0in, 3.3in)}]
    {\Huge \bfseries C++14};
\node[scale=2.6, text=YellowOrange, at={(0in, 2.3in)}]
    {\Huge \bfseries FAQs};

% Author
\node[scale=1.1, text=Gainsboro, at={(0in, -3.0in)}]
    {\Huge \bfseries \emph{Chandra Shekhar Kumar}};


\end{scope}

  
\end{tikzpicture}



\end{document}